%%%%%%%%%%%%%%%%%%%%%%%%%%%%%%%%%%%%%%%%%%%%%%%%%%%%%%%%%%%%%%%%%%%%%%%%%%%%%
% 26/05/2010
% edited by Bill Lampos
%
% Feel free to use (copy) the structure (latex formatting source code)
% but not the content of this document.
%
%%%%%%%%%%%%%%%%%%%%%%%%%%%%%%%%%%%%%%%%%%%%%%%%%%%%%%%%%%%%%%%%%%%%%%%%%%%%%
\documentclass[compress,blue]{beamer}
\mode<presentation>

\usetheme{Warsaw}
% other themes: AnnArbor, Antibes, Bergen, Berkeley, Berlin, Boadilla, boxes, CambridgeUS, Copenhagen, Darmstadt, default, Dresden, Frankfurt, Goettingen,
% Hannover, Ilmenau, JuanLesPins, Luebeck, Madrid, Maloe, Marburg, Montpellier, PaloAlto, Pittsburg, Rochester, Singapore, Szeged, classic

%\usecolortheme{whale}
% color themes: albatross, beaver, beetle, crane, default, dolphin, dov, fly, lily, orchid, rose, seagull, seahorse, sidebartab, structure, whale, wolverine

%\usefonttheme{serif}
% font themes: default, professionalfonts, serif, structurebold, structureitalicserif, structuresmallcapsserif

% pdf is displayed in full screen mode automatically
%\hypersetup{pdfpagemode=FullScreen}

% define your own colours:
\definecolor{Red}{rgb}{1,0,0}
\definecolor{Blue}{rgb}{0,0,1}
\definecolor{Green}{rgb}{0,1,0}
\definecolor{magenta}{rgb}{1,0,.6}
\definecolor{lightblue}{rgb}{0,.5,1}
\definecolor{lightpurple}{rgb}{.6,.4,1}
\definecolor{gold}{rgb}{.6,.5,0}
\definecolor{orange}{rgb}{1,0.4,0}
\definecolor{hotpink}{rgb}{1,0,0.5}
\definecolor{newcolor2}{rgb}{.5,.3,.5}
\definecolor{newcolor}{rgb}{0,.3,1}
\definecolor{newcolor3}{rgb}{1,0,.35}
\definecolor{darkgreen1}{rgb}{0, .35, 0}
\definecolor{darkgreen}{rgb}{0, .6, 0}
\definecolor{darkred}{rgb}{.75,0,0}

\xdefinecolor{olive}{cmyk}{0.64,0,0.95,0.4}
\xdefinecolor{purpleish}{cmyk}{0.75,0.75,0,0}

% \usepackage{beamerinnertheme_______}
% inner themes include circles, default, inmargin, rectangles, rounded

%\usepackage{beamerouterthemesmoothbars}
% outer themes include default, infolines, miniframes, shadow, sidebar, smoothbars, smoothtree, split, tree

\useoutertheme[subsection=false]{smoothbars}

% to have the same footer on all slides
%\setbeamertemplate{footline}[text line]{xxx xxx xxx}
%\setbeamertemplate{footline}[text line]{} % or empty footer

% include packages
\usepackage[utf8]{inputenc}
\usepackage[english]{babel}
\usepackage{subfigure}
\usepackage{multicol}
\usepackage{amsmath}
\usepackage{epsfig}
\usepackage{graphicx}
\usepackage[all,knot]{xy}
\xyoption{arc}
\usepackage{url}
\usepackage{multimedia}
\usepackage{hyperref}
\usepackage{setspace}

\title{A Comparison of Approaches to\\ Large-Scale Data Analysis}
\subtitle{}
\author{Magnus Kirø}
\date{\scriptsize Norwegian University of Science and Technology \\ \vspace{.10cm} \today }


% here begins the presentation content. 
\begin{document}

\frame{
	\titlepage
}

\section[Outline]{}
\subsection*{}

%\subsection{Overview}
\frame{\frametitle{Presentation Goals}
The purpose of paper is to consider MapReduce and a regular Database Management Systems for large-scale data analysis. 

\vspace{0.25cm}
\begin{enumerate}
	\item \textbf{Parallel DBMS} and \textbf{MR}, two approaches to large-scale data analysis.  
\vspace{0.25cm}
	\item Thee \textbf{Architectural Elements} of DBMS and MR.
\vspace{0.25cm}
	\item Outlinei of \textbf{Benchmark Results} and \textbf{Best Practice} for large-scale data analysis.
\end{enumerate}
}

\frame{\tableofcontents}


\section{Two Approaches} % - to large scale data analysis. 
\subsection{DBMS}
\frame{\frametitle{Database Management System}
{\footnotesize
\begin{tabbing}
Title: \hspace{1.25cm} \= \textbf{Tracking the flu pandemic by monitoring the Social Web}\\
Authors:               \> V. Lampos and N. Cristianini\\
Submitted to:          \> IAPR Cognitive Information Processing 2010 (accepted)
\end{tabbing}
}
\begin{itemize}
\item Twitter and Health Protection Agency data for weeks 26-49, 2009 (on average 160,000 tweets collected per day geolocated in 54 urban centres in the UK)
\item Frequency of \textbf{41 flu related words} (markers) in Twitter corpus had a correlation of $>$\textbf{80\%} with the HPA flu rates in all UK regions
\item Learn a better list of weighted markers \textbf{automatically}:
    \begin{itemize}
    \item Generate a list of candidate markers (1560 words taken from flu related web pages)
    \item Use \textbf{LASSO} for feature selection
    \end{itemize}
\end{itemize}
}

\frame{\frametitle{DBMS - }

\textbf{Validation} schemes:
    \begin{enumerate}
    \item Train on one region, validate regularisation parameter on another, test on the remaining regions (for all possible combinations)

    \begin{table}[!t]
    \tiny
    \renewcommand{\arraystretch}{1}
    \centering
    \begin{tabular}{cccccc}
    \hline
    Train/Validate (regions) & \textbf{A} & \textbf{B}      & \textbf{C} & \textbf{D}       & \textbf{E}     \\\hline
    \textbf{A}     & -          & 0.9594          & 0.9375     & 0.9348           & 0.9297         \\%\hline
    \textbf{B}     & 0.9455     & -               & 0.9476     & 0.9267           & 0.9003         \\%\hline
    \textbf{C}     & 0.9154     & 0.9513          & -          & 0.8188           & 0.908          \\%\hline
    \textbf{D}     & 0.9463     & 0.9459          & 0.9424     & -                & 0.9337         \\%\hline
    \textbf{E}     & 0.8798     & 0.9506          & 0.9455     & 0.8935           & -              \\\hline
    &              &            &\multicolumn{2}{c}{Total Avg.}                   & \textbf{0.9256}\\\cline{4-6}
    \end{tabular}
    \end{table}

    \begin{spacing}{0.5}{
    \textbf{97 selected words}\tiny : lung, unwel, temperatur, like, headach, season, unusu, chronic, child, dai, appetit, stai, symptom, spread, diarrhoea, start, muscl, weaken, immun, feel, liver, plenti, antivir, follow,
    sore, peopl, nation, small, pandem, pregnant, thermomet, bed, loss, heart, mention, condit, ...
    }
    \end{spacing}

    \item Aggregate data from all regions, test on weeks 28 and 41 (2009) and train using the rest of the data set
    \end{enumerate}
}

\frame{\frametitle{DBMS - hei}
\begin{enumerate}
\item Inferred vs Official flu rate in North England\\
\centering \includegraphics[scale=.5]{figures/Lasso_Inference_regionC_NEng.pdf}

\item Inferred vs Official rates in all regions (aggregated data set)\\
\centering \includegraphics[scale=.5]{figures/Lasso_Inference_Aggregate_IMPROVED.pdf}

\end{enumerate}
}

\subsection{Map Reduce}
\frame{\frametitle{MR - }
{\footnotesize
\begin{tabbing}
Title: \hspace{1.25cm} \= \textbf{Flu detector - Tracking epidemics on Twitter}\\
Authors:               \> V. Lampos, T. De Bie, and N. Cristianini\\
Submitted to:          \> ECML PKDD 2010 Demos (under review)
\end{tabbing}
}

\begin{itemize}
\item Extending and making more robust the methodology of P1
\item Larger data sets (bigger time series) and more (2675) \href{http://geopatterns.enm.bris.ac.uk/epidemics/files/candidate_features.txt}{candidate features}
\item Select a list of features (markers) using BoLASSO (bootstrap version of LASSO)
\item Then learn weights of those markers via linear least squares regression
\item Stricter evaluation of the methodology - \href{http://geopatterns.enm.bris.ac.uk/epidemics/twitter-flu-eval.php}{\textbf{Available online}}
\item Put all this into practice and come up with the \href{http://geopatterns.enm.bris.ac.uk/epidemics/}{\textbf{Flu detector}}
\end{itemize}
}

\section{Architectural Elements} % trade-offs in architectures. 
\subsection{Schema Support}
\frame{\frametitle{Schema Support}
\begin{enumerate}
\item ... (content omitted)
\item ... (content omitted)
\item ... (content omitted)
\item ... (content omitted)
\end{enumerate}
}

\subsection{Indexing}
\frame{\frametitle{Indexing}
\begin{enumerate}
\item ... (content omitted)
\item ... (content omitted)
\item ... (content omitted)
\item ... (content omitted)
\end{enumerate}
}

\subsection{Programming Model}
\frame{\frametitle{Programming Model}
\begin{enumerate}
\item ... (content omitted)
\item ... (content omitted)
\item ... (content omitted)
\item ... (content omitted)
\end{enumerate}
}

\subsection{Data Distribution}
\frame{\frametitle{Data Distribution}
\begin{enumerate}
\item ... (content omitted)
\item ... (content omitted)
\item ... (content omitted)
\item ... (content omitted)
\end{enumerate}
}

\subsection{Execution Strategy}
\frame{\frametitle{Execution Strategy}
\begin{enumerate}
\item ... (content omitted)
\item ... (content omitted)
\item ... (content omitted)
\item ... (content omitted)
\end{enumerate}
}

\subsection{Flexibility}
\frame{\frametitle{Flexibility}
\begin{enumerate}
\item ... (content omitted)
\item ... (content omitted)
\item ... (content omitted)
\item ... (content omitted)
\end{enumerate}
}

\subsection{Fault Tolerance}
\frame{\frametitle{Fault Tolerance}
\begin{enumerate}
\item ... (content omitted)
\item ... (content omitted)
\item ... (content omitted)
\item ... (content omitted)
\end{enumerate}
}

\section{Results}
\subsection{Benchmark Results}
\frame{\frametitle{A more time specific tentative plan}

}

\subsection{Best Practice}
\frame{\frametitle{A more time specific tentative plan}

}

\section*{}
\frame{
    \begin{center}
        \huge
        This is the last slide.\\
        \vspace{1cm}
        Any questions?
    \end{center}
}

\end{document} 
